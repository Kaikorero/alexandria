\documentclass[a4paper, 12pt]{scrarticle}

\usepackage{ifxetex,fontspec}
\setmainfont{FreeSerif}
\defaultfontfeatures{Ligatures=TeX}
%\usepackage{xeCJK}
\usepackage{chronology} % Für einen Zeitstrahl


%Titelinformationen
%--------------------------------------
\author{Adrian Kinzig}
\date{\today}
%--------------------------------------

\usepackage{colortbl}
\usepackage{longtable}

\usepackage{polyglossia}
    \setdefaultlanguage{german}
    \setotherlanguage[variant = ancient]{greek}
    \newfontfamily\greekfont[Script = Greek]{Aegean}

	%Literatur
	%--------------------------------------
	\usepackage[backend=biber,
            style=verbose-trad3, 
            mincrossrefs=1,
            sortlocale=auto]{biblatex}
        \addbibresource{~/Uni/standard.bib}
	%--------------------------------------


\title{
        \begin{normalsize}
Quellenübersicht für die Hausarbeit:
        \end{normalsize} \\
Alexandria Eschate\\
        \begin{normalsize}
Motive für die Gründung der östlichsten Stadt
        \end{normalsize}}

%--------------------------
%HIER GEHTS LOS
%--------------------------

\begin{document}

% Sammlung der Quellen
\maketitle

% Legende
	\noindent
	Legende:\vfill

	Das Symbol \textbf{--} zeigt das Fehlen der Stadt in der Quelle an,
	\textbf{?} deutet entweder an,
	dass die Quelle nicht vorliegt und mir daher noch nicht klar ist,
	ob die Stadt erwähnt wird,
	oder aber die Identität ist unsicher.

	Farbe:\vfill
	farblos                = Quelle nur gelesen/ nur aus Zusammenfassung bekannt\vfill
	\colorbox{red}{ }      = Keine Übersetzung, aber bereits digital erfasst\vfill
	\colorbox{orange}{ }   = Übersetzt, Quellenkritik unvollständig\vfill
	\colorbox{yellow}{ }   = Forschungsliteratur, Übersetzung oder Edition veraltet\vfill
	\colorbox{green}{ }    = Fertig\vfill

%Beginn der Tabelle für die Autoren

\begin{longtable}{lllll}
        %Tabellenkopf, auf jeder Seite wiederholt
	Autor & Stelle & Datierung & Stand \\
        \hline
\endhead

	%----------------AUTOREN-GRIECHISCH-------------
        \textbf{Griechische Autoren}&
        &
        &
        &
        \cellcolor{white}\\


%%%%%%%%%%%%%%%%%%%%%%%%%%%%%%%%%%%%%
	%----------------AUTOREN-LATEIN-------------
        \textbf{Lateinische Autoren}&
        &
        &
        &
        \cellcolor{white}\\

	%Curtius Rufus
        Quintus Curtius Rufus&
	Historiae Alexandri 7 V,27 (23) - (29) &
        &
        \cellcolor{white}\\
  
\end{longtable}




%Beginn der Tabelle für die Inschriften%%%%%%%%%%%%%%%%
\begin{longtable}{lllll}

        %Tabellenkopf, auf jeder Seite wiederholt
        Inschrift & Fundort & Datierung & Stand \\
        \hline
\endhead


	%----------------INSCHRIFTEN------------
        \textbf{Inschriften gesamt} &
        &
        &
        \cellcolor{white}\\

	%---------IEOG-------------------
		%---------IEOG-391------------------
		\citetitle[391]{ieog}&
		Alexandria Eschate?&
		328 v. Chr.&
		\cellcolor{white} ? \\

		%---------IEOG-392------------------
		\citetitle[392]{ieog}&
		Didyma (Milet)?&
		306 v. Chr?&
		\cellcolor{white} ?\\

		%---------IEOG-393------------------
		\citetitle[393]{ieog}&
		Didyma (Milet)&
		300/299 v. Chr.&
		\cellcolor{white}\\

		%---------IEOG-394------------------
		\citetitle[394]{ieog}&
		Didyma (Milet)&
		299/298 v. Chr.&
		\cellcolor{white}\\

\end{longtable}

\printbibliography

\end{document}

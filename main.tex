%! TEX TS-program = xelatex
\documentclass[a4paper]{scrreprt}

%Codierung Xetex
%-------------------------------------
\usepackage{pifont} % Wirklich schlimme Sonderzeichen...
\usepackage{ucharclasses, fontspec}
        \setmainfont[SmallCapsFont={GFSDidot-Regular},
                  SmallCapsFeatures={Letters=SmallCaps}]{Noto Serif}
        \setsansfont{Noto Sans}
        \setmonofont{Noto Sans Mono}

%Fallback
\newfontfamily{\defaultfont}{Noto Serif}
\newfontfamily{\akrophon}{Quivira}
\newfontfamily\hebrewfont{Noto Serif Hebrew}[Script=Hebrew]

\setTransitionFrom{AncientGreekNumbers}{\defaultfont} % UTF-Block-Name
\setTransitionTo{AncientGreekNumbers}{\akrophon}
%-------------------------------------

%Abbildungen
%--------------------------------------
\usepackage{graphicx}
\graphicspath{ {./img} }
\usepackage{wrapfig}    % Für das Einbetten von Bildern
\usepackage{chronology} % Für einen Zeitstrahl
\usepackage{rotating}   % Zum rotieren von Bildern
%--------------------------------------

%Lorem Ipsum und Kommentare
%--------------------------------------
\usepackage{blindtext}
\usepackage{verbatim}
%--------------------------------------

%Mehrspaltigkeit, Tabellen und anderes Layout
%--------------------------------------
\usepackage{paracol}            % Mehrere Spalten mit unterschiedlichem Inhalt
	\footnotelayout{m}                  % Fußnoten bei Paracol merged mit den anderen!
\usepackage[protrusion]{microtype}   % Verbesserter Randausgleich
	\emergencystretch=1.5em         % Bessere Zeilenumbrüche (Fehler werden weiter ausgegeben)
\usepackage{verse}              %Umgebung für Gedichte
    	\setlength{\leftmargini}{0em}       % Linker Rand Abstand (Nummerierung)
    	\verselinenumfont{\tiny\rmfamily}   % Größe und Aussehen der Nummerierung
\usepackage[modulo]{lineno}     % Zeilennummerierung
\usepackage[all]{nowidow}       % Keine Schusterjungen und Hurenkinder
\usepackage{longtable}          % Tabellen
\usepackage{lscape}             % Querformat
%--------------------------------------

%Web
%--------------------------------------
\usepackage[colorlinks, 
            linkcolor = black, 
            citecolor = black, 
            filecolor = black, 
            urlcolor = blue]{hyperref}
%--------------------------------------

%Sprachen
%--------------------------------------
\usepackage{polyglossia}        % Verwendete Sprachen
        \setdefaultlanguage{german}
        \setotherlanguage[variant = ancient]{greek}
%	\setotherlanguage[variant = modern]{latin}
%        \setotherlanguage{hebrew}
\usepackage{csquotes}           % Anführungszeichen an Sprache anpassen
%--------------------------------------

%Titelinformationen (einfach)
%--------------------------------------
\title{Zwischen Kult und Mythos\\
Diktynna, Britomartis, Aphaia -- Eine Göttin, drei Kulte?}
\author{Adrian Kinzig}
\date{\today}
%----------------------------

%Literatur
%--------------------------------------
\usepackage[backend=biber,
            style=verbose-trad3, 
            mincrossrefs=1,
            sortlocale=auto]{biblatex}
        \addbibresource{~/Uni/standard.bib}
%--------------------------------------

%Quellen
%--------------------------------------
\defbibheading{quellen}
    {\section*{Literarische Quellen}}
\defbibheading{bildquelle}
    {\section*{Bild- und Archäologische Quellen }}
\defbibheading{sekundär}
    {\section*{Literatur}}
%--------------------------------------

%Eigene Befehle
%--------------------------------------
    %Besondere Markierungen
  \newcommand{\fw}[1]{\textit{#1}} %Fachwörter kursiv, /fw{x}

    %Befehle für die Zeilennummerierung in Versen
  \newcommand{\verselinks}[2]{
          \poemlines{5}
          \verselinenumbersleft
          \setlength{\vrightskip}{-7cm}
          \setverselinenums{#1}{#2}}

  \newcommand{\verserechts}[2]{
          \poemlines{5}
          \setverselinenums{#1}{#2}}

    %Befehle für Zwischenüberschrfiten
  \newcommand{\ittitle}[1]{
           \begin{center}
           \textit{#1}
           \end{center}}

%--------------------------------------

\begin{document}

\thispagestyle{empty}
\begin{titlepage}

\titlehead{
                \begin{flushright}
Georg-August-Universität Göttingen\\
Philosophische Fakultät -- Althistorisches Seminar\\
Masterstudiengang Antike Kulturen\\
Modul M.ALTER.12: Antike Politikgeschichte\\
Dozentin: Dr. Dorit Engster\\
                \end{flushright}
}

\subject{Modulprüfung}
\title{Alexandria Eschate}
\subtitle{Motive für die Gründung der östlichsten Stadt}
\author{Adrian Kinzig}
\date{\today}
\publishers{
21801478\\
Mitteldorfstraße 12\\
37083 Göttingen\\
adrian.kinzig@stud.uni-goettingen.de\\
}

\maketitle
\end{titlepage}

%\newpage 
%\thispagestyle{empty}
%\quad 
%\newpage
\tableofcontents
\thispagestyle{empty}
\clearpage


\thispagestyle{empty}
\begin{titlepage}

\titlehead{
                \begin{flushright}
Georg-August-Universität Göttingen\\
Philosophische Fakultät -- Althistorisches Seminar\\
Masterstudiengang Antike Kulturen\\
Modul M.ALTER.12: Antike Politikgeschichte\\
Dozentin: Dr. Dorit Engster\\
                \end{flushright}
}

\subject{Modulprüfung}
\title{Alexandria Eschate}
\subtitle{Motive für die Gründung der östlichsten Stadt}
\author{Adrian Kinzig}
\date{\today}
\publishers{
21801478\\
Mitteldorfstraße 12\\
37083 Göttingen\\
adrian.kinzig@stud.uni-goettingen.de\\
}

\maketitle
\end{titlepage}

%\newpage 
%\thispagestyle{empty}
%\quad 
%\newpage
\tableofcontents
\thispagestyle{empty}
\clearpage


\input{inhalt/titelseite.tex}

\input{inhalt/einleitung/info.tex}
\input{inhalt/hauptteil/info.tex}
\input{inhalt/schluss/info.tex}

Vorläufige Gliederung

\begin{description}
	\item[538 v. Chr.] 
		Kyros II. erobert Baktrien und Sogdien.
		Er gründet zum Schutz gegen die Saken und Massageten mehrere Festungen,
			von denen eine vermutlich Kyropolis ist.
	\item[494 v. Chr.] 
		Dareios bekämpft die Skythen und schlägt danach den ionischen Aufstand nieder. (Herodot)
		Möglicherweise siedelt er Ionier an der skythischen Grenze an (Curt. Ruf.)
	\item[329 v. Chr.]
		Alexander findet in Baktrien und Sogdien Griechen vor,
			die ihm teilweise nicht wohl gesonnen waren.
		Er gründet Alexandria Eschate. (Curt. Ruf.)
		Eine der Städte,
			die er bei Auseinandersetzungen zerstört,
			ist Kyropolis. (Curt. Ruf.)


\end{description}

\input{inhalt/titelseite.tex}

\input{inhalt/einleitung/info.tex}
\input{inhalt/hauptteil/info.tex}
\input{inhalt/schluss/info.tex}


Vorläufige Gliederung

\begin{description}
	\item[538 v. Chr.] 
		Kyros II. erobert Baktrien und Sogdien.
		Er gründet zum Schutz gegen die Saken und Massageten mehrere Festungen,
			von denen eine vermutlich Kyropolis ist.
	\item[494 v. Chr.] 
		Dareios bekämpft die Skythen und schlägt danach den ionischen Aufstand nieder. (Herodot)
		Möglicherweise siedelt er Ionier an der skythischen Grenze an (Curt. Ruf.)
	\item[329 v. Chr.]
		Alexander findet in Baktrien und Sogdien Griechen vor,
			die ihm teilweise nicht wohl gesonnen waren.
		Er gründet Alexandria Eschate. (Curt. Ruf.)
		Eine der Städte,
			die er bei Auseinandersetzungen zerstört,
			ist Kyropolis. (Curt. Ruf.)


\end{description}

\thispagestyle{empty}
\begin{titlepage}

\titlehead{
                \begin{flushright}
Georg-August-Universität Göttingen\\
Philosophische Fakultät -- Althistorisches Seminar\\
Masterstudiengang Antike Kulturen\\
Modul M.ALTER.12: Antike Politikgeschichte\\
Dozentin: Dr. Dorit Engster\\
                \end{flushright}
}

\subject{Modulprüfung}
\title{Alexandria Eschate}
\subtitle{Motive für die Gründung der östlichsten Stadt}
\author{Adrian Kinzig}
\date{\today}
\publishers{
21801478\\
Mitteldorfstraße 12\\
37083 Göttingen\\
adrian.kinzig@stud.uni-goettingen.de\\
}

\maketitle
\end{titlepage}

%\newpage 
%\thispagestyle{empty}
%\quad 
%\newpage
\tableofcontents
\thispagestyle{empty}
\clearpage


\input{inhalt/titelseite.tex}

\input{inhalt/einleitung/info.tex}
\input{inhalt/hauptteil/info.tex}
\input{inhalt/schluss/info.tex}

Vorläufige Gliederung

\begin{description}
	\item[538 v. Chr.] 
		Kyros II. erobert Baktrien und Sogdien.
		Er gründet zum Schutz gegen die Saken und Massageten mehrere Festungen,
			von denen eine vermutlich Kyropolis ist.
	\item[494 v. Chr.] 
		Dareios bekämpft die Skythen und schlägt danach den ionischen Aufstand nieder. (Herodot)
		Möglicherweise siedelt er Ionier an der skythischen Grenze an (Curt. Ruf.)
	\item[329 v. Chr.]
		Alexander findet in Baktrien und Sogdien Griechen vor,
			die ihm teilweise nicht wohl gesonnen waren.
		Er gründet Alexandria Eschate. (Curt. Ruf.)
		Eine der Städte,
			die er bei Auseinandersetzungen zerstört,
			ist Kyropolis. (Curt. Ruf.)


\end{description}

\input{inhalt/titelseite.tex}

\input{inhalt/einleitung/info.tex}
\input{inhalt/hauptteil/info.tex}
\input{inhalt/schluss/info.tex}


    % Hier ist der Inhalt

%\chapter{Quellen und Literaturverzeichnis}
%\printbibliography[keyword=quelle, notkeyword=grabung, heading=quellen]
%\printbibliography[keyword=grabung,  heading=bildquelle]
%\printbibliography[notkeyword=quelle, notkeyword=grabung, heading=sekundär]

\end{document}
